\begin{figure}[!ht]
    \centering
    %\vspace{0.5cm}

    \subfloat[]{
    \begin{tikzpicture}[>=triangle 45, scale=1.5]
    
    % Ideal Oscillator
    \draw[->, thick] (0,0) -- (2.4,0) node[right] {$\omega$};
    \draw[->, thick] (1.2,0) -- (1.2,3) node[above] {$S_v(\omega)$};
    %\draw[thick] (1,0) -- (1,1.8);
    \node at (1.2,-0.3) {$\omega_c$};
    \node at (1.2,3.7) {Идеални осцилатор};

    \end{tikzpicture}
    \label{fig:osc:noise_1a}
    }
    \hspace{1.5cm}
    \subfloat[]{
    \begin{tikzpicture}[>=triangle 45, scale=1.5]
    
    % Practical Oscillator
    \draw[->, thick] (0,0) -- (4,0) node[right] {$\omega$};
    \draw[->, thick] (2,0) -- (2,3) node[above] {$S_v(\omega)$};
    
    % Drawing the curve
    \draw[thick] plot[domain=0.5:3.5, samples=100] (\x, {0.2 + 2.5*exp(-4*(\x-2)^2)});
    
    % Vertical line at wc
    % \draw[dashed, thick] (2,0) -- (2,2);

    \draw[-] (2.8,0) -- (2.8,0.8); 
    \draw[-] (2.85,0) -- (2.85,0.8); 

    % Delta omega and bandwidth labels
    \draw[<->] (2,0.7) -- node[midway, below] {$\Delta \omega$} (2.8,0.7);
    \draw[->] (3.4,0.6) -- node[near start, above, anchor=south west] {Опсег од 1\,Hz} (2.87,0.1);

    % Omega_c label
    \node at (2,-0.3) {$\omega_c$};
    \node at (2,3.7) {Реални осцилатор};
    
    \end{tikzpicture}
    \label{fig:osc:noise_1b}
    }
    %\vspace{0.5cm}
    \caption{Излазни спектар (а) идеалног и (б) реалног осцилатора.}
    \label{fig:osc:noise_1}
\end{figure}
