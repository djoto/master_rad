% FLL block diagram
\begin{figure*}[ht]
	\centering
	\begin{tikzpicture}[thick, scale=0.53, every node/.style={scale=0.45}]
		% Grid
		%\draw[step=1cm, gray, very thin] (0,0) grid (40,10);
		% Setup
		%\tikzset{every node/.style={align=center}};
		%\tikzstyle{every node}=[draw]
		% Beginning
		\draw                         (0  ,0) -- (0  ,5);
		\draw [-]                     (0  ,5) -- (1  ,5);
		\draw [-,  \CtrlColor]        (1  ,5) -- (1.5,5);
		\draw [->, \CtrlPreprocColor] (1.5,5) -- (1.94  ,5);
		% Preprocessing stage (grey counter + gra2bin conversion + synchronization)
		\node (GREYCNT)  at (2.8,5) [draw, \CtrlPreprocColor, minimum size=2cm, minimum height=2.4cm, align=center]{ГРЕЈЕВ  \\ БРОЈАЧ \\[5pt] \framebox{1,3,2,6}};
		\node (CONVSYNC) at (5.8,5) [draw, \CtrlPreprocColor, minimum size=2cm, minimum height=2.4cm, align=center]{СИНХР. \\ БИНАРНИ \\ ПРЕТВАРАЧ \\[5pt] \framebox{$\rightleftarrows$}};
		\draw [->, \CtrlPreprocColor] (GREYCNT) -- (CONVSYNC);
		\draw [dashed, \CtrlPreprocColor, fill=\CtrlPreprocColor, fill opacity=0.1, text opacity=1] (1.5,3.5) rectangle (7.5,6.5) node[midway, above=0.85]{УПРАВЉАЧКА ПРЕДОБРАДА};
		% Splitter/demultiplexer
		\node (DEMUX) at (8.5,5) [draw, \CtrlColor, trapezium, trapezium stretches=true, minimum width=7cm, minimum height=1cm, rotate=90]{РАЗДЈЕЛНИК \engl{SPLITTER}};
		\draw [-,  \CtrlPreprocColor] (CONVSYNC) -- (7.5,5);
		\draw [->, \CtrlColor]           (7.5,5) -- (DEMUX);
		% Bang-bang FLL control branch
		\draw [-,    \CtrlColor]  (9  ,6.5) -- ( 9.5,6.5);
		\draw [->, \BBctrlColor] ( 9.5,6.5) -- (10  ,6.5);
		\node (CMP) at (11.4,6.5) [draw, \BBctrlColor, minimum size=2cm, minimum height=2.4cm, align=center]{КОМПАРАТОР \\[5pt] \framebox{$\geq$}};
		\node (CNT) at (14.2,6.5) [draw, \BBctrlColor, minimum size=2cm, minimum height=2.4cm, align=center]{БРОЈАЧ \\[5pt] \framebox{1,2,3}};
		\draw [->, \BBctrlColor]      (CMP) --      (CNT);
		\draw [-,  \BBctrlColor]      (CNT) -- (15.5,6.5);
		\draw [->,   \CtrlColor] (15.5,6.5) -- (16  ,6.5);
		\draw [dashed, \BBctrlColor, fill=\BBctrlColor, fill opacity=0.1, text opacity=1] (9.5,5) rectangle (15.5,8) node[midway, above=0.85]{ДВОСТЕПЕНИ КОНТРОЛЕР};
		% PID FLL control branch
		\draw [->, \CtrlColor] ( 9,3.5) -- (11,3.5);
		\node (PID) at (12.5,3.5) [draw, \PIDctrlColor, fill=\PIDctrlColor, fill opacity=0.1, text opacity=1, minimum width=3.5cm, minimum height=2.4cm, align=center]{PID \\ РЕГУЛАТОР \\[5pt]};
		\draw [\PIDctrlColor] plot[smooth] coordinates {(11.75,2.7) (12.0,3.2) (12.5,3.1) (13.25,3.1)};
		\draw [\PIDctrlColor, fill=\PIDctrlColor, fill opacity=0.1] (11,2.5) rectangle (14,4.5) node[midway, below=0.6, opacity=1] {PID КОНТРОЛЕР};
		\draw [->, \CtrlColor] (14,3.5) -- (16,3.5);
		% Control logic box
		\draw [dashed, \CtrlColor, fill=\CtrlColor, fill opacity=0.1, text opacity=1] (1,1) rectangle (21,9) node[midway, above=2.15]{УПРАВЉАЧКА ЛОГИКА};
		% Multiplexer
		\node (MUX) at (16.5,5) [draw, \CtrlColor, trapezium, trapezium stretches=true, minimum width=7cm, minimum height=1cm, rotate=-90]{МУЛТИПЛЕКСЕР};
		% Control Decoder
		\node (CTRLDEC) at (19,5) [draw, \CtrlDecColor, fill=\CtrlDecColor, fill opacity=0.1, text opacity=1, minimum size=2cm, minimum height=2.4cm, align=center]{УПРАВЉАЧКИ \\ ДЕКОДЕР};
		\draw [->, \CtrlColor] (MUX)  -- (CTRLDEC);
		% DCO
		\node (DCO) at (23,5) [draw, \DCOColor, fill=\DCOColor, fill opacity=0.1, text opacity=1, minimum size=2.4cm, minimum height=2.4cm, align=center]{DCO \\[5pt]};
		\draw [\DCOColor](22.8,4.7) sin (22.9,4.8) cos (23.0,4.7) sin (23.1,4.6) cos (23.2,4.7);
		\draw [\DCOColor](23.0,4.7) circle (0.3);
		\draw [-, \CtrlColor] (CTRLDEC) -- (21,5);
		\draw [->]                     (21,5) --  (DCO);
		% End
		\draw  (DCO) -- (25,5);
		\draw (25,5) -- (25,0);
		\draw (25,0) --  (0,0);
		
        \draw (-1,-1) rectangle (26,10) node[midway, above=2.95]{\textbf{ПРЕДЛОЖЕНА ДИГИТАЛНА ФРЕКВЕНЦИЈСКИ ЗАТВОРЕНА ПЕТЉА}};	
        \draw [thick, ->] (-2.5,9) -- (-1,9) node[midway, above=0.1] {\textbf{RSTN}};
        \draw [thick, ->] (-2.5,8) -- (-1,8) node[midway, above=0.1] {\textbf{FREF}};        	
        \draw [thick, ->] (-2.5,7) -- (-1,7) node[midway, above=0.1] {\textbf{FMUL}};
        \draw [thick, ->] (-2.5,6) -- (-1,6) node[midway, above=0.1] {\textbf{CTRL}};
        \draw (-2.0,6.825) --  (-1.5,7.175);
        \draw (-2.0,5.825) --  (-1.5,6.175);
        %\draw [-]  (25,5) -- (26,5);
        \draw [thick, ->] (26,8) -- (27.5,8) node[midway, above=0.1] {\textbf{FOUT}};
        \draw [thick, ->] (26,7) -- (27.5,7) node[midway, above=0.1] {\textbf{LOCK}};
        
		
	\end{tikzpicture}
	\caption{Блок дијаграм дигиталне фреквенцијски затворене петље састављене од: блока управљачке логике (лијево) и дигитално контролисаног осцилатора (десно).}
	\label{FLL block diagram}
\end{figure*}
