\begin{figure}[!ht]
    \centering
    \tikzset{flipflop DQ/.style={flipflop,
             flipflop def={t1=D, t6=Q, c3=1, font=\normalsize}
    }}
    \begin{tikzpicture}

    \draw
    (-4,0) node[draw, rectangle, dashed, text width=2.5cm, align=center](gray){Грејев\\бројач\\у подручју\\такта \DCO-a}

    % First flip-flop
    (gray.east) --++ (1.5,0) node[flipflop DQ, anchor=pin 1] (ff1) {}
    
    % Second flip-flop
    (ff1.pin 6) --++ (1.5,0) node[flipflop DQ, anchor=pin 1] (ff2) {}
    
    (ff2.pin 6) --++ (1.5,0) node[draw, rectangle, dashed, text width=2.5cm, align=center, anchor=west](bin){Бинарни претварач у подручју референтног такта}

    (-2.5,-3) node(ref){} --++ (1.2,0) node[](circ1){} -| (ff1.pin 3)
    (circ1) -| (ff2.pin 3)

    (ref.center) node[draw, rectangle, dashed, text width=2.5cm, align=center, anchor=east]{Референтни такт}
    ;
    \draw[dashed] (gray.north east) ++ (1,0) node[](start){} --++ (7,0) --++ (0,-4.5) -| (start.center);
    \draw (start.north east) ++ (1.8,0.2) node[anchor=west]{Синхронизатор};
    \end{tikzpicture}
    \caption{Синхронизатор са два флип-флопа.}
    \label{fig:control_preprocessing:sync}
\end{figure}
