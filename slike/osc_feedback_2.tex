\begin{figure}[!ht]
    \centering
    \vspace{0.5cm}
    \begin{tikzpicture}[auto, >=triangle 45]
    
    % Op-Amp
    \draw (0,0) node[op amp] (opamp) {};

    \node [draw, circle, minimum size=0.15cm, inner sep=0,  fill=black, right of=opamp, node distance=1.5cm] (out) {};
    \node [draw, circle, minimum size=0.15cm, right of=out, inner sep=0] (circvout) {};
    
    \draw (opamp.+) node[draw, left, circle, minimum size=0.15cm, inner sep=0, node distance=2cm] (circvin) {};
    \node [left of=circvin, node distance=0.4cm] (vin) {$V_{in}$};
    \node [right of=circvout, node distance=0.5cm] (vout) {$V_{out}$};
    \draw (opamp.out) -- (out) -- (circvout);
    \draw (out) |- ++(0,1.3) -| (opamp.-);
    
    \draw (-2.9,-0.5) sin (-2.7,-0.1) cos (-2.5,-0.5) sin (-2.3,-0.9) cos (-2.1,-0.5);
    \draw (3.5,0) sin (3.7,0.4) cos (3.9,0) sin (4.1,-0.4) cos (4.3,0);

    \end{tikzpicture}
    %\vspace{0.5cm}
    \caption{Реализација повратне спреге коришћењем операционог појачавача.}
    \label{osc_feedback_2}
\end{figure}
