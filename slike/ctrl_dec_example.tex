\begin{figure}[!ht]
    \centering
    \begin{tikzpicture}[>=triangle 45]

    % Параметри
    \def\rows{17}
    \def\cols{15}
    \def\size{0.2} % Величина кружића
    \def\spacing{0.6} % Мањи размаци између кружића
    \def\arrowspacing{0.4} % Размак између стрелица и бројева
    \def\leftcolumngap{2.5} % Већи размак између двије колоне стрелица

    % Петља за генерисање кружића
    \foreach \r in {1,...,\rows} {
        \foreach \c in {1,...,\cols} {
            % Испуњеност бојом (прве три врсте су црне, четврта има само 5 црних кружића)
            \ifnum \r=4
                \ifnum \c<6
                    \fill (\c * \spacing, -\r * \spacing) circle (\size);
                \else
                    \draw (\c * \spacing, -\r * \spacing) circle (\size);
                \fi
            \else
                \ifnum \r<4
                    \fill (\c * \spacing, -\r * \spacing) circle (\size);
                \else
                    \draw (\c * \spacing, -\r * \spacing) circle (\size);
                \fi
            \fi
        }
    }

    % Стрелице изнад прве врсте и бројеви
    \foreach \c in {1,...,\cols} {
        \draw[->, fill=black] (\c * \spacing, \arrowspacing) -- (\c * \spacing, -0.1);
%        \node at (\c * \spacing, \arrowspacing + 0.25) {0};
        \ifnum \c<6
            \node at (\c * \spacing, \arrowspacing + 0.25) {1};
%            \fill (\c * \spacing, -\r * \spacing) circle (\size);
        \else
            \node at (\c * \spacing, \arrowspacing + 0.25) {0};
%            \draw (\c * \spacing, -\r * \spacing) circle (\size);
        \fi
    }
    \node at (8 * \spacing, \arrowspacing + 0.8) {\textit{Column Select}};

    % Прва колона стрелица и бројеви лијево од прве колоне
    \foreach \r in {1,...,\rows} {
        \draw[->, fill=black] (-0.5, -\r * \spacing) -- (0, -\r * \spacing);
        \ifnum \r=4
            \node at (-0.7, -\r * \spacing) {1};
        \else
            \node at (-0.7, -\r * \spacing) {0};
        \fi
    }
    \node at (-1.2, -9 * \spacing) [rotate=90] {\textit{Row Select}};

    % Друга колона стрелица и бројеви лијево од прве колоне, са већим размаком
    %\foreach \r in {1,...,\rows} {
    %    \draw[->, fill=black] (-\leftcolumngap, -\r * \spacing) -- (-2, -\r * \spacing);
    %    \ifnum \r<4
    %        \node at (-\leftcolumngap - 0.2, -\r * \spacing) {1};
    %    \else
    %        \node at (-\leftcolumngap - 0.2, -\r * \spacing) {0};
    %    \fi
    %}
    %\node at (-3.2, -9 * \spacing) [rotate=90] {\textit{Row On}};

    % Колона стрелица и бројеви десно од задње колоне
    \foreach \r in {1,...,\rows} {
        \draw[->, fill=black] (10.1, -\r * \spacing) -- (9.6, -\r * \spacing);
        \ifnum \r<4
            \node at (10.3, -\r * \spacing) {1};
        \else
            \node at (10.3, -\r * \spacing) {0};
        \fi
    }
    \node at (10.7, -9 * \spacing) [rotate=90] {\textit{Row On}};

    \end{tikzpicture}
    \caption{Упрошћен примјер начина управљања врстама и колонама матрице прекидача.}
    \label{fig:ctrl_dec:example}
\end{figure}
