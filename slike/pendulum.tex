\begin{figure}[!ht]
    \centering
    \vspace{0.5cm}
    \begin{tikzpicture}

	% Position 1
	\node at (0, 0) {};
	\draw[thick] (0,0) -- (-2,-2);
	\draw[dashed] (0,0) -- (0,-2.84);
	\draw[fill=white] (0,0) circle (2pt);
	\draw[fill=white, thick] (-2,-2) circle (6pt);
	\node[align=center] at (-2,-2.8) {потенцијална\\енергија};
	\node at (0,0.5) {положај 1};

	% Position 2
	\begin{scope}[shift={(3,0)}]
	\draw[thick] (0,0) -- (0,-3);
	\draw[fill=white] (0,0) circle (2pt);
	\draw[fill=white, thick] (0,-3) circle (6pt);
	\node[align=center] at (1.2,-2.7) {кинетичка\\енергија};
	\draw[<-,thick] (1,-3.5) arc[start angle=-60, end angle=-120, radius=2];
	\node at (0,0.5) {положај 2};
	\end{scope}

	% Position 3
	\begin{scope}[shift={(6,0)}]
	\draw[thick] (0,0) -- (2,-2);
	\draw[dashed] (0,0) -- (0,-2.84);
	\draw[fill=white] (0,0) circle (2pt);
	\draw[fill=white, thick] (2,-2) circle (6pt);
	\node[align=center] at (2.2,-2.8) {потенцијална\\енергија};
	\node at (0,0.5) {положај 3};
	\end{scope}

    \end{tikzpicture}
    %\vspace{0.5cm}
    \caption{Клатно које дјелује као осцилаторни систем~\cite{Razavi:PLL_CMOS_2020}.}
    \label{pendulum}
\end{figure}
