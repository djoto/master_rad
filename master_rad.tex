\documentclass[master]{finthesis}

\usepackage{subcaption}
\usepackage{tikz}
\usetikzlibrary{through, patterns}

\addbibresource{master_rad.bib}

\title{Синтетизабилна дигитална фреквенцијски затворена петља са широким подешавањем до 640\texorpdfstring{\,}{ }MHz у 130\texorpdfstring{\,}{ }nm CMOS технологији}
\author{Ђорђе С. Гачић}
\studid{408/2022}
\advisor{проф. др Владимир М. Миловановић}
\advisorfull{Др Владимир М. Миловановић,\\ванредни професор}

\studprog{Електротехника и рачунарство}
% \module{Модул}
\course{Напредно машинско учење}
\date{\today}
% \date{ГГГГ-ММ-ДД}

\committee{др Шћепан Шћекић}
\committee{доц. др Жарко Попара}

\studentshould{% У оквиру овог рада кандидат треба да...
}

%\titlepagebib{greenwade93}
%\titlepagebib{pythonsite}
\titlepagebib{Staszewski:FREQUENCY_SYNTHESIZER_CMOS_2005}

\thesisapplicationfile{slike/prijava}

\abstracten{Frequency-locked loops (FLLs) represent a viable way of generating a range of frequencies from a single reference frequency by using a negative feedback electronic control system that compares the frequency of a controlled oscillator to the reference one. A digital synthesizable FLL is designed in 130\,nm CMOS technology for a target frequency of up to 640\,MHz. It employs a wide-tuning range digitally controlled oscillator (DCO) assembled from tri-state inverters in the form of a matrix. The FLL can optionally use a bang-bang or a soft-programmable standard proportional-integral-derivative (PID) controller to regulate the feedback loop. Its design practically minimizes metastability occurrence. The proposed digital FLL occupies 100\,\textmu m $\times$ 330\,\textmu m and consumes 3.5\,mW in typical operating conditions. The reference clock is 16\,MHz, and the output oscillation frequency is set to 640\,MHz, while the achieved frequency resolution is 2.8\,MHz.}
\keywordsen{Frequency-locked loop, digitally controlled oscillator, clock generator, synthesizable, CMOS technology, PID controller, metastability}

\abstractsr{Фреквенцијски затворене петље (енгл. \textit{frequency-locked loops - FLLs}) представљају одржив начин генерисања опсега фреквенција из једне референтне фреквенције коришћењем негативно повратног електронског система управљања који пореди фреквенцију контролисаног осцилатора са поменутом референтном фреквенцијом. Дигитално синтетизабилан FLL је дизајниран у 130\,nm технологији за циљану фреквенцију до 640\,MHz. Он погони дигитално контролисани осцилатор (енгл. \textit{digitally-controlled oscillator - DCO}) са широким подешавањем опсега који се састоји од тростатаичких инвертора у облику матрице. FLL може произвољно користити тзв. \textit{bang-bang} контролер или дјелимично програмирани стандардни пропорционално-интегрално-диференцијални (енгл. \textit{proportional-integral-derivative - PID}) контролер за управљање негативном петљом. Такав дизајн у пракси минимизује појаву метастабилности. Предложени дигитални FLL заузима 100\,\textmu m $\times$ 330\,\textmu m простора и троши 3.5\,mW у уобичајеним условима рада. Референтни такт је 16\,MHz, а излазна фреквенција осциловања је подешена на 640\,MHz, док постигнута резолуција фреквенције износи 2.8\,MHz.}
\keywordssr{Фреквенцијски затворена петља, дигитално контролисани осцилатор, генератор такта, синтетизабилност, CMOS технологија, PID контролер, метастабилност}

% Semantic markup. Look it up!
%\newcommand{\cmd}[2][]{\texttt{\textbackslash #2#1\{\ldots\}}} % Well, this one is useful beyond just semantics...
%\newcommand{\env}[1]{\texttt{#1}}
%\newcommand{\pkg}[1]{\texttt{#1}}
%\newcommand{\prog}[1]{\texttt{#1}}

\begin{document}

\maketitle

\tableofcontents

\makeabstract

\section{Увод}
У данашње вријеме, фазно затворена петља (енгл. \textit{phase-locked loop - PLL)} и петља са затвореним кашњењем (енгл. \textit{delay-locked loop - DLL}) представљају свеприсутне блокове у дизајну чипова. Безброј примјена самих чипова захтјевају или генератор такта или синтетизатор фреквенције, што подразумјева уградњу неког од поменутих блокова унутар система који се пројектује. Главна улога таквог блока у дизајну је да генерише стабилан и прецизан излазни сигнал чија је фаза подесива у односу на фазу улазног сигнала, самим тим одржавајући везу између улазне и излазне фреквенције. Међутим, чак и веома сложени системи често захтјевају генератор такта, који само множи улазну фреквенцију без да посебно води рачуна о фази такта или апсолутном подрхтавању (енгл. \textit{jitter}). У таквим примјенама, потребна и довољна је само фреквенцијски затворена петља (енгл. \textit{frequency-locked loop - FLL}) да би се испунили тражени захтјеви.

По дефиницији, FLL је негативно повратни управљачки систем који закључава фреквенцију излазног сигнала на предвиђену циљану фреквенцију. У принципу, непрастано управља фреквенцијом осцилатора на аутоматски начин све док излазна фреквенција на достигне циљану вриједност, након чега се та вриједност фреквенција одржава на излазу. Постоје многи начини имплементације FLL-а~\cite{Ali:9097205}. Штавише, FLL као интегрисано коло може спадати у двије групе: дигитални и аналогни FLL. Иако је очигледан недостатак првих максимална фреквенција и њена резолуција, они посједују многе друге предности наспрам њихових аналогних супарника. Они заузимају мање простора, истичу се већом отпорношћу на промјене процесних углова, напона и температуре (енгл. \textit{process-voltage-temperature - PVT}), лако су употребљиви у различитим технологијама, и стога омогућавају поновну употребу, већу прилагодљивост, једноставнију методологију тока пројектовања, као и брже циклусе пројектовања. Узимајући у обзир све претходно поменуто, испоставља се да је у општем случају боље ићи ка развоју дигиталног FLL-a кад год спецификација архитектуре система то дозвољава. Дакле, фокус овог рада је пројектовати и унаприједити једноставне али моћне синтетизабилне дигиталне блокове чипа.

Овај рад конкретно предлаже синтетизабилан дигитални FLL сличан предложеном у литератури~\cite{Musa:6644316}, са побољшаном брзином закључавања FLL-а~\cite{Deng:6891375} и смањеним ризиком од метастабилности. Осцилатор је састављен од тростатичких инвертора и заснован на прстенастом DCO-у из литературе~\cite{Tierno:4443210} измјењен додавањем независног напона напајања DCO-а са мјењачима нивоа (енгл. \textit{level shifters}) и употребом петостепене~\cite{Rylyakov:4523284} умјесто тростепене толологије прстена осцилатора.

Остатак рада укључује додатна поглавља. Поглавље 2 описује предложени дигитални FLL на системском нивоу и нивоу блокова и кола уз детаљна теоријска разматрања. Поглавље 3 пружа увид у имплементацију и добијене резултате симулација, такође уз теоријска разматрања појава које су од значаја за рад читавог система. Коначно, поглавље 4 закључује рад и наговјештава могућности даљег рада на побољшању и проширењу система.


\section{Разрада}

\section{Закључак}

\makebibliography

\end{document}
